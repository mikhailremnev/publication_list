\documentclass[10pt, a4paper]{article}
\usepackage[T2A]{fontenc}
\usepackage[utf8]{inputenc}
\usepackage[english,russian]{babel}

\usepackage[usenames,dvipsnames,svgnames,table]{xcolor}
\usepackage[colorlinks=true, linkcolor=Maroon]{hyperref}

\usepackage[height=25cm, a4paper, hmargin={1.35cm, 1.35cm}, vmargin={1.35cm, 1.35cm}]{geometry}

\usepackage{longtable}
\usepackage{tabularx}

% https://tex.stackexchange.com/a/5020
% (add new column type)
\usepackage{array}
% C{1cm} -- centered column with paragraphs
\newcolumntype{C}[1]{>{\centering\arraybackslash\hspace{0pt}}p{#1}}
% L{1cm} -- left-aligned column with paragraphs
%           (not sure why, we need \raggedright for left alignment)
\newcolumntype{L}[1]{>{\raggedright\arraybackslash\hspace{0pt}}p{#1}}

\title{Публикации}

\author{}

\date{\today}

\begin{document}

%%%%%%%%%%%%%%%%%%%%%%%%%%%%%%%%%%%%%%%%%%%%%%%%
% Примечания:
% Список составляется по разделам в хронологической последовательности
% публикации работ по сквозной нумерации:
% а)научные работы;
% б)авторские свидетельства, дипломы, патенты, лицензии, информационные карты,
% алгоритмы, проекты;
% в)учебно-методические работы;
% В графе 2 (Наименование) приводится полное наименование работы (тема) с
% уточнением в скобках вида публикации: монография, статья, тезисы, отчеты по НИР,
% прошедшие депонирование; учебник, учебное пособие, руководство, учебно-
% методическая разработка и другие. При необходимости указывается, на каком языке
% опубликована работа.
% В графе 3 (Форма работы) указывается соответствующая форма объективного
% существования работы: печатная, рукописная, аудиовизуальная, компьютерная и др.
% Дипломы и авторские свидетельства, патенты, лицензии, информационные карты,
% алгоритмы, проекты не характеризуются (делается прочерк).
% В графе 4 (Выходные данные) конкретизируются место и время публикации
% (издательство, номер или серия периодического издания; указывается тематика,
% категория, место и год проведения научных и методических конференций, симпозиумов,
% семинаров и съездов, в материалах которых содержатся тезисы доклада (выступления,
% сообщения): международные, всероссийские, региональные, отраслевые,
% межотраслевые, краевые, областные, межвузовские, вузовские (научно-педагогического
% состава, молодых специалистов, студентов и т.д.); номер диплома на открытие,
% авторского свидетельства на изобретение, свидетельства на промышленный образец,
% дата их выдачи; номер патента и дата выдачи, номер регистрации и дата оформления
% лицензий, информационных карт, алгоритмов, проектов.
% Все данные приводятся в соответствии с правилами библиографического
% описания литературы.
% В графе 5 (Объем) указывается количество печатных листов (п.л.) или страниц
% (с.) публикаций (дробно: в числителе – общий объем, в знаменателе – объем,
% принадлежащий соискателю).
% В графе 6 (Соавторы) перечисляются фамилии и инициалы соавторов в порядке
% их участия в работе. Из состава больших авторских коллективов приводятся фамилии
% первых пяти человек, после чего проставляется «и др., всего ___ человек».
%%%%%%%%%%%%%%%%%%%%%%%%%%%%%%%%%%%%%%%%%%%%%%%%

\begin{flushright}Форма №16\end{flushright}

\begin{center}
    \textbf{СПИСОК}\\
    \textbf{опубликованных и приравненных к ним}\\
    \textbf{научных и учебно-методических работ}\\
    \vspace{1em}
    \textbf{РЕМНЕВА МИХАИЛА АНАТОЛЬЕВИЧА}
\end{center}

\begin{center}

\newcounter{publicationID}
\setcounter{publicationID}{0}

\begin{tabular}{|C{10mm}|p{35mm}|p{12mm}|p{30mm}|p{20mm}|p{29mm}|}
\hline
№ п/п & Наименование
работы,  её вид,
системы цитирования & Форма
работы & Выходные
данные & Объём
в п.л. или 
с. & Соавторы \\
\hline
1 & 2 & 3 & 4 & 5 & 6 \\
\hline
\end{tabular}

\begin{flushleft}
\textbf{а) научные работы:}
\end{flushleft}

\begin{longtable}{|C{10mm}|L{35mm}|p{12mm}|L{30mm}|p{20mm}|p{29mm}|}
% \caption{A simple longtable example}\\
\hline
% № п/п & Наименование
% работы,  её вид,
% системы цитирования & Форма
% работы & Выходные
% данные & Объём
% в п.л. или 
% с. & Соавторы \\
% \hline
\endfirsthead
% \multicolumn{4}{c}%
% {\tablename\ \thetable\ -- \textit{Continued from previous page}} \\
% \hline
% \textbf{First entry} & \textbf{Second entry} & \textbf{Third entry} & \textbf{Fourth entry} \\
\hline
\endhead
\hline
% \multicolumn{4}{r}{\textit{Continued on next page}} \\
\endfoot
\hline
\endlastfoot
1 & Charged particle identification with the liquid xenon calorimeter of the CMD-3 detector, WoS, Scopus & печ. & NUCL INSTRUM METH A 2021 г. т. 1015 165761 & 17/0.362 & Ivanov, Fedotovich, Akhmetshin, Amirkhanov, Anisenkov и др., всего 47 чел. \\
2 & Development of negative muonium ion source for muon acceleration, WoS, Scopus & печ. & PHYS REV ACCEL BEAMS 2021 г. т. 24 033403 & 9/0.562 & Kitamura, Bae, Choi, Fukao, Iinuma и др., всего 16 чел. \\
3 & Search for the Process e(+)e(-)-> D*(2007)(0) with the CMD-3 Detector, WoS & печ. & PHYS ATOM NUCL+ 2020 г. т. 83 с. 954-957 & 4/0.068 & Shemyakin, Akhmetshin, Amirkhanov, Anisenkov, Aulchenko и др., всего 59 чел. \\
4 & Study of the process e(+)e(-) -> KSK +/-pi(-/+) with CMD-3 detector, WoS, Scopus & печ. & PHYS SCRIPTA 2020 г. т. 95 104002 & 4/0.068 & Uskov, Fedotovich, Ivanov, Akhmetshin, Amirkhanov и др., всего 59 чел. \\
5 & Study of the process e(+)e(-) -> (KSKS0)-K-0 pi(+)pi(-) in the c.m. energy range 1.6-2.0 GeV with the CMD-3 detector, WoS, Scopus & печ. & PHYS LETT B 2020 г. т. 804 135380 & 8/0.140 & Akhmetshin, Amirkhanov, Anisenkov, Aulchenko, Banzarov и др., всего 57 чел. \\
6 & Overview of PID options for experiments at the Super Charm-Tau Factory, WoS, Scopus & тез. & J INSTRUM 2020 г. т. 15 C04032 & 12/0.706 & Barnyakov, Barnyakov, Belozyorova, Boborovnikov, Buzykaev и др., всего 17 чел. \\
7 & Particle identification system for the Super Charm-Tau Factory at Novosibirsk, WoS, Scopus & тез. & NUCL INSTRUM METH A 2020 г. т. 958 162352 & 4/0.211 & Barnyakov, Barnyakov, Boborovnikov, Buzykaev, Bykov и др., всего 19 чел. \\
8 & Measurement of the e(+)e(-)-> eta pi(+)pi(-) cross section with the CMD-3 detector at the VEPP-2000 collider, WoS, Scopus & печ. & J HIGH ENERGY PHYS 2020 г. 112 & 24/0.400 & Gribanov, Popov, Akhmetshin, Amirkhanov, Anisenkov и др., всего 60 чел. \\
9 & Recent results from the CMD-3 detector, WoS, Scopus & тез. & NUCL PART PHYS P 2020 г. т. 309 с. 99-102 & 4/0.070 & Lukin, Akhmetshin, Amirkhanov, Anisenkov, Aulchenko и др., всего 57 чел. \\
10 & Overview of the CMD-3 recent results, WoS, Scopus & тез. & J PHYS CONF SER 2020 г. т. 1526 012009 & 5/0.089 & Ryzhenenkov, Akhmetshin, Amirkhanov, Anisenkov, Aulchenko и др., всего 56 чел. \\
11 & Study of the process e(+)e(-)-> K+K-eta with the CMD-3 detector at the VEPP-2000 collider, WoS, Scopus & печ. & PHYS LETT B 2019 г. т. 798 134946 & 10/0.179 & Ivanov, Fedotovich, Akhmetshin, Amirkhanov, Anisenkov и др., всего 56 чел. \\

\end{longtable}
\end{center}

\vspace{2em}

%%% \begin{tabularx}{.9\textwidth}{lXl}
%%%     Соискатель & & И. И. Иванов \\
%%%     & & \\
%%%     & & \\
%%%     \textbf{Список  верен:} & & \\
%%%     & & \\
%%%     & & \\
%%%     Ученый секретарь & & А. В. ---------- \\
%%%     & & \\
%%%     & & \\
%%%     & & (дата)
%%% \end{tabularx}

\end{document}
