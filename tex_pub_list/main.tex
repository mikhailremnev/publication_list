\documentclass[10pt, a4paper]{article}
\usepackage[T2A]{fontenc}
\usepackage[utf8]{inputenc}
\usepackage[english,russian]{babel}

\usepackage[usenames,dvipsnames,svgnames,table]{xcolor}
\usepackage[colorlinks=true, linkcolor=Maroon]{hyperref}

\usepackage[height=25cm, a4paper, hmargin={1.35cm, 1.35cm}, vmargin={1.35cm, 1.35cm}]{geometry}

\usepackage{longtable}
\usepackage{tabularx}

% https://tex.stackexchange.com/a/5020
% (add new column type)
\usepackage{array}
% C{1cm} -- centered column with paragraphs
\newcolumntype{C}[1]{>{\centering\arraybackslash\hspace{0pt}}p{#1}}
% L{1cm} -- left-aligned column with paragraphs
%           (not sure why, we need \raggedright for left alignment)
\newcolumntype{L}[1]{>{\raggedright\arraybackslash\hspace{0pt}}p{#1}}

\title{Публикации}

\author{}

\date{\today}

\begin{document}

%%%%%%%%%%%%%%%%%%%%%%%%%%%%%%%%%%%%%%%%%%%%%%%%
% Примечания:
% Список составляется по разделам в хронологической последовательности
% публикации работ по сквозной нумерации:
% а)научные работы;
% б)авторские свидетельства, дипломы, патенты, лицензии, информационные карты,
% алгоритмы, проекты;
% в)учебно-методические работы;
% В графе 2 (Наименование) приводится полное наименование работы (тема) с
% уточнением в скобках вида публикации: монография, статья, тезисы, отчеты по НИР,
% прошедшие депонирование; учебник, учебное пособие, руководство, учебно-
% методическая разработка и другие. При необходимости указывается, на каком языке
% опубликована работа.
% В графе 3 (Форма работы) указывается соответствующая форма объективного
% существования работы: печатная, рукописная, аудиовизуальная, компьютерная и др.
% Дипломы и авторские свидетельства, патенты, лицензии, информационные карты,
% алгоритмы, проекты не характеризуются (делается прочерк).
% В графе 4 (Выходные данные) конкретизируются место и время публикации
% (издательство, номер или серия периодического издания; указывается тематика,
% категория, место и год проведения научных и методических конференций, симпозиумов,
% семинаров и съездов, в материалах которых содержатся тезисы доклада (выступления,
% сообщения): международные, всероссийские, региональные, отраслевые,
% межотраслевые, краевые, областные, межвузовские, вузовские (научно-педагогического
% состава, молодых специалистов, студентов и т.д.); номер диплома на открытие,
% авторского свидетельства на изобретение, свидетельства на промышленный образец,
% дата их выдачи; номер патента и дата выдачи, номер регистрации и дата оформления
% лицензий, информационных карт, алгоритмов, проектов.
% Все данные приводятся в соответствии с правилами библиографического
% описания литературы.
% В графе 5 (Объем) указывается количество печатных листов (п.л.) или страниц
% (с.) публикаций (дробно: в числителе – общий объем, в знаменателе – объем,
% принадлежащий соискателю).
% В графе 6 (Соавторы) перечисляются фамилии и инициалы соавторов в порядке
% их участия в работе. Из состава больших авторских коллективов приводятся фамилии
% первых пяти человек, после чего проставляется «и др., всего ___ человек».
%%%%%%%%%%%%%%%%%%%%%%%%%%%%%%%%%%%%%%%%%%%%%%%%

\begin{flushright}Форма №16\end{flushright}

\begin{center}
    \textbf{СПИСОК}\\
    \textbf{опубликованных и приравненных к ним}\\
    \textbf{научных и учебно-методических работ}\\
    \vspace{1em}
    \textbf{РЕМНЕВА МИХАИЛА АНАТОЛЬЕВИЧА}
\end{center}

\begin{center}

\newcounter{publicationID}
\setcounter{publicationID}{0}

\begin{tabular}{|C{10mm}|p{35mm}|p{12mm}|p{30mm}|p{20mm}|p{29mm}|}
\hline
№ п/п & Наименование
работы,  её вид,
системы цитирования & Форма
работы & Выходные
данные & Объём
в п.л. или 
с. & Соавторы \\
\hline
1 & 2 & 3 & 4 & 5 & 6 \\
\hline
\end{tabular}

\begin{flushleft}
\textbf{а) научные работы:}
\end{flushleft}

\begin{longtable}{|C{10mm}|L{35mm}|p{12mm}|L{30mm}|p{20mm}|p{29mm}|}
% \caption{A simple longtable example}\\
\hline
% № п/п & Наименование
% работы,  её вид,
% системы цитирования & Форма
% работы & Выходные
% данные & Объём
% в п.л. или 
% с. & Соавторы \\
% \hline
\endfirsthead
% \multicolumn{4}{c}%
% {\tablename\ \thetable\ -- \textit{Continued from previous page}} \\
% \hline
% \textbf{First entry} & \textbf{Second entry} & \textbf{Third entry} & \textbf{Fourth entry} \\
\hline
\endhead
\hline
% \multicolumn{4}{r}{\textit{Continued on next page}} \\
\endfoot
\hline
\endlastfoot
\input{table_body}
\end{longtable}
\end{center}

\vspace{2em}

%%% \begin{tabularx}{.9\textwidth}{lXl}
%%%     Соискатель & & И. И. Иванов \\
%%%     & & \\
%%%     & & \\
%%%     \textbf{Список  верен:} & & \\
%%%     & & \\
%%%     & & \\
%%%     Ученый секретарь & & А. В. ---------- \\
%%%     & & \\
%%%     & & \\
%%%     & & (дата)
%%% \end{tabularx}

\end{document}
